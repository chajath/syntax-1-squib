\documentclass{article}
\usepackage[utf8]{inputenc}
\usepackage[margin=1in]{geometry}
\usepackage{kotex}
\usepackage{natbib}
\usepackage{setspace}
\usepackage{gb4e}
\bibliographystyle{chicago}
\doublespacing

\title{
`-tul' Plural Marker in Korean\\
\large{Squib Proposal}
}

\author{In-Ho Yi}
\date{2018}

\begin{document}

\maketitle

\section{Introduction}

The purpose of this squib is to explore characteristics of \textit{`-tul'} plural marker in the Korean language. What is interesting about this plural marker is that it is thought to be copying itself from the canonical position of the suffix to the subject, to other non-canonical elements such as adverbs, locatives and connecting particles of complex verbs (\cite{SSONG75}).

\begin{exe}

\ex
\glll 사람들_i이 식사들_e 맛있게들_e 하고들_e 있다 \\
     salam-tul_i-i siksa-tul_e masiss-key-tul_e ha-ko-tul_e iss-ta \\
     people-PL_i-NORM meal-PL_e delicious-ly-PL_e do-CONN-PL_e exist-DECL \\

\trans `People are having delicious meals'
\end{exe}

In this example, tul$_i$ is an Intrinsic Plural Marker (IPM), whereas tul$_e$ is an Extrinsic Plural Marker (EPM), following conventions from \cite{SONG97}. IPM denotes the plurality of the very term that hosts it, whereas EPMs denote the plurality of the subject of the sentence.

In this squib, we will first generalize the distribution of EPMs. We will also examine the various theoretical considerations resulting from the EPM phenomena.

\section{Background}

EPMs in Korean have long been thought to be an optional particle attached to signal the plurality of the subject in the sentence (\cite{SSONG75}). Under this framework, \textit{`-tul'} originates from the subject, and copies itself to other elements, even when the subject is covert or dropped. One advantage of this approach is that the original `-tul' is in the c-commanding position of the copied `-tul's, and the schema fits nicely with Binding Condition A (\cite{KIM06}).

\cite{SONG97}, on the other hand, thinks that the EPMs have their own meaning and pragmatic functions, such as denoting distributivity and focus. This view will be critically examined in this squib, along with some of the judgments presented in \cite{SONG97} that I find hard to accept.

\section{Preliminary Data Gathering}

\begin{exe}

\ex
\glll 식사(들) 맛있게(들) 했니 \\
     siksa(-tul) masiss-key(-tul)  ha-yss-ni \\
     meal(-PL)    delicious-ly(-PL) do-PST-Q \\
\trans `Did you have a delicious meal'

\end{exe}

\begin{exe}

\ex \begin{xlist}

\ex
\glll 왜 나한테들 그래 \\ 
     way na-hanthey-tul kul-ay \\
     why 1.SG-DAT-PL like-Q \\
\trans `Why are they like that to me'

\ex
\gll na-hanthey way-tul kul-ay \\
     1.SG-DAT why-PL like-Q \\

\ex[?]{
\gll na-hanthey-tul way-tul kul-ay \\
     1.SG-DAT-PL why-PL like-Q \\
}

\ex[?]{way-tul na-hanthey-tul kul-ay}

\ex[??]{na-hanthey-tul way kul-ay}

\ex[??]{way-tul na-hanthey kul-ay}

\end{xlist}

\end{exe}


\begin{exe}

\ex \begin{xlist}

\ex
\gll ese(-tul) cip-ey(-tul)    ka-la \\
     hurry(-PL) home-LOC(-PL) go-IMP \\
\trans `Go home now!'

\ex 
\gll 집에(들) 어서(들) 가라 \\
     cip-ey(-tul) ese(-tul) ka-la \\

\end{xlist}

\end{exe}

In (2)-(4), adverbs, locatives, and datives can host EPMs. In (3) c-f, I find repeating EPMs and EPM in distance more unnatural than a single, most immediate EPM to the main verb. In (4), on the other hand, the order and the frequency of EPMs don't seem to affect judgment. This point hasn't been raised in other literature (\cite{SONG97}'s focus theory of the EPMs predicts different judgments) and will be examined more carefully in this squib.

\begin{exe}

\ex
\glll 아주(?들) 웃기고(들) 있다 \\ 
     acwu(-?tul) wuski-ko(-tul) iss-ta \\
     very(-PL) being.funny-CONN(-PL) exist-DECL \\
\trans `They are being very funny'

\ex
\glll 많이(들) 먹어 \\ 
     manhi(-tul) mek-e \\
     many-PL eat-IMP \\
\trans `Eat a lot' (`Enjoy your meal')

\end{exe}

Some adverbs are more natural to host EPMs than others. Many speakers would reject \textit{acwu-tul} in (5) outright, while others would only accept it in informal and online settings. It will be interesting to examine what the determining factor is.

\begin{exe}

\ex 
\glll ``나는 못 하겠다" 라고(들) 말한다 \\
     ``na-nun   mos ha-keyss-ta"   lako(-tul) malha-n-ta \\
     ``1.SG-TOP not do-FUT-DECL"   COMP(-PL)  say-PRES-DECL \\
\trans `They say ``I won't do it"'

\end{exe}

It seems to be well-established that EPMs don't appear inside embedded and relative clauses, while complementizers can host EPMs.

\section{Research Directions}

\subsection{EPM and subjecthood}

If EPMs are the copies of IPMs, then it could be argued that the EPMs are the stranded copies when the subject was moving from an adjoined position to the verb to the specifier of IP position in the sentence, analogous to how floating Q is analyzed (\cite{MCC97}).

\subsection{EPM and verbal feature selection}

Subjects are often dropped from sentences in the Korean language. Furthermore, EPMs seem to appear more in the imperative mood. Rather than having the plural subject copy out its plural marker before disappearing, I would like to have a verb, through feature selection, emit those plural markers to various elements in the sentence, including the subject. It could be argued that those plural markers are copied when the plural verbal phrase is merged with other elements. Both IPM and EPM are generated through the same mechanism, and we might be able to do away with the IPM/EPM dichotomy.

\section{Conclusion}

EPM is an intriguing phenomenon. Given that the wide variety of elements can host EPM, care should be taken in generalizing the distribution of EPM. Once we have the generalization, other theoretical concerns will be explored in light of the existence of EPMs.

\singlespacing
\medskip
\bibliography{refs}

\end{document}
